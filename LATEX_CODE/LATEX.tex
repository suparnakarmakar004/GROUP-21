\documentclass{article}
\usepackage{graphicx} % Package for including images
\usepackage{amsmath}
\usepackage{enumitem}
\begin{document}

\title{My First LaTeX Document}
\author{Name -Biswajit Naskar \\
Department - BCA \\
Roll-30001223014 \\
Registration no. - 233001010488}
\date{August 30, 2024}

\maketitle

\begin{center}
\includegraphics[width=0.5\textwidth]{Biswajit.jpg}
\end{center}

\section{Introduction}
\subsection{About Me}
I am a Biswajit Naskar.I live in west Bengal.Learning on MAKAUT.

\subsection{Interests and Hobbies}
\begin{itemize}
\item Thing 1: e-sport.
\item Thing 2: Cricket.
    \begin{itemize}
    \item My 1st hobbies.
        \begin{itemize}
        \item I love play e-sport Game.
        \item Refrace me.
        \end{itemize}
    \item My 2nd hobbies.
        \begin{itemize}
        \item Healdy Body.
        \item My inspiration MS DHONI.
        \end{itemize}
    \end{itemize}
\end{itemize}

\subsection{Favorite Quotations}
\begin{enumerate}
\item MS DHONI
\item RATAN TATA
\end{enumerate}



\section{Mathematics}

\subsection{Mathematics and Me}

Reflection on Experiences with Mathematics

What do I like about mathematics?
I appreciate the logical structure and problem-solving aspect of mathematics. It provides a framework to understand the world around us, from the simplest numerical relationships to complex theories. The satisfaction that comes from solving a challenging problem is unparalleled; it feels like putting together the pieces of a puzzle. Additionally, mathematics has a universal language that transcends cultural and linguistic barriers, which is intriguing.


How far would I like to take my study of mathematics?
I am eager to explore mathematics further, possibly delving into higher-level topics such as calculus, linear algebra, and statistics. I’d like to apply mathematical concepts to real-world problems, particularly in fields like data science, economics, or engineering. Ultimately, I envision pursuing mathematics at a more advanced level, potentially considering a degree or specialization that allows me to leverage math in practical applications.


What have I enjoyed learning this year in mathematics?
This year, I have particularly enjoyed learning about algebra and its applications. The ability to manipulate equations and understand relationships between variables has been enlightening. Additionally, exploring geometry and its principles has been fascinating, especially when applying them to solve problems related to shapes, areas, and volumes. Learning about mathematical modeling and how to represent real-world situations with mathematical expressions has been particularly rewarding.


What have I found the most challenging?
The most challenging aspect of my mathematical studies this year has been tackling more abstract concepts, such as functions and their transformations. At times, it can be difficult to visualize how changes in a function's parameters affect its graph and behavior. Additionally, understanding proofs and the logical reasoning behind mathematical concepts has posed challenges, as it requires a different way of thinking compared to simply solving numerical problems. However, these challenges have also been opportunities for growth, pushing me to develop a deeper understanding of the subject.

\subsection{Mathematical Notation}

Choose a four-digit number which you will use to practice typesetting mathematical expressions. Typeset everything below, including all text just as you see it, substituting your four-digit number in place of the sample number 1972 wherever it occurs (use appropriate values when simplifying the equation in 4(b)).

\subsubsection{Superscripts, subscripts, and Greek letters}

\begin{enumerate}[label=(\alph*)]
\item $21^{72}$
\item $3^{9^{72}}$
\item $21_{72}$
\item $2_{97_{2}}$
\item $2472\pi$
\item $\cos~\theta$
\item $\tan^{-1}(2.472)$
\item $\log_{24}72$
\item $\ln 2472$
\item $e^{2.472}$
\item $0<x\le2472$
\item $y\ge2472$
\end{enumerate}

\subsubsection{Roots, fractions, and displaystyle}

\begin{enumerate}[label=(\alph*)]
\item $\sqrt{1972}$
\item $\sqrt[10]{72}$
\item $\displaystyle \frac{19}{72}$
\item $\displaystyle \frac{1}{9+\frac{7}{2}}$
\item $\displaystyle \sqrt{\frac{19}{72}}$
\end{enumerate}

\section{Tables and Equation Arrays}

\subsection{Tables and Equation Arrays}

\begin{enumerate}
\item (a)
    \begin{tabular}{c|c|c|c|c}
        $x$ & 1 & 9 & 9 & 7 \\ \hline
        $f(x)$ & 1 & 9 & 9 & 7 \\
    \end{tabular}

    (b)
    \begin{align*}
        3 + 4 + 9 + 7 &= 23 \\
        27 - 12 &= 15
    \end{align*}
\end{enumerate}

\section{Functions and Formulas}

\begin{enumerate}[label=(\alph*)]
\item The quadratic formula:
    \[x = \frac{-b \pm \sqrt{b^2 - 4ac}}{2a}\]

\item The function $f(x) = \left(x + \frac{2}{4}\right)^2 - \frac{7}{2}$ has domain $D_f: (-\infty, \infty)$ and range $R_f: \left(-\frac{7}{2}, \infty\right)$.

\item Definition of a Derivative:
    \[\lim_{h \to 0} \frac{f(x+h) - f(x)}{h} = f'(x)\]

\item Chain Rule: $[f(g(x))]' = f'(g(x)) \cdot g'(x)$

\item $\frac{d^2y}{dx^2} = f''(x)$

\item $\int \sec^2 x \, dx = \tan x + C$

\item $\int e^{2x} \, dx = \frac{1}{2} e^{2x} + C$

\item Fundamental Theorem of Calculus, Part 1: $\int_a^b f'(x) \, dx = f(b) - f(a)$

\item Fundamental Theorem of Calculus, Part 2: $\frac{d}{dx} \int_a^{g(x)} f(t) \, dt = f(g(x)) \cdot g'(x)$

\item Euler's Method: $y_1 = y_0 + hF(x_0, y_0)$ where $h$ is the step size, and $F(x, y) = \frac{dy}{dx}$.

\item $a_n = \left\{1972, \frac{1972}{2}, \frac{1972}{2^2}, \frac{1972}{2^3}, \dots, \frac{1972}{2^n}\right\}$ represents a geometric sequence.

\item $S_n = \sum_{n=1}^\infty \frac{2472}{2^n}$ is a convergent geometric series since $|r| = \left|\frac{1}{2}\right| < 1$.

\item Taylor Series: $\sum_{n=0}^\infty \frac{f^{(n)}(c)}{n!} (x-c)^n$

\item Velocity Vector: $\vec{r}(t) = x'(t)\vec{i} + y'(t)\vec{j} = \left\langle \frac{dx}{dt}, \frac{dy}{dt} \right\rangle$

\item Area of Polar Curve: $A = \frac{1}{2} \int_a^\beta r^2 \, d\theta$

\end{enumerate}






\end{document}
